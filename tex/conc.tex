\documentclass{standalone}

\begin{document}
	The CENA method computes forward error in linear algorithms and makes corrections to offset for rounding error introduced by elementary operations in finite precision arithmetic. This paper presents the application of the CENA compensations on Newton's method for one-dimensional zero finding of functions of increasing degree but same root. Experimental results show that while the CENA leads to lower relative error in general, it is not equivalent to doubling the working precision. Thus, the study shows that the CENA method may not be suitable for critical numerical applications while further testing may be needed as outlined. Instead, a different direction for the CENA method is identified as it shows promise for highlighting issues related to rounding errors and compensatory frameworks related to such errors.
\end{document}