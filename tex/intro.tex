\documentclass{standalone}

\begin{document}
	Rounding errors are a ``fact of life'' when dealing with any computation in finite precision arithmetic. Famous examples of arithmetic errors include the many jokes inspiring Intel division bug \cite{sharangpani1994statistical} and the subtle accumulating rounding errors suddenly revealing itself in the Vancouver stock exchange \cite{mackintosh2015}. More serious examples of rounding errors leading to catastrophic consequences include the explosion of the spacecraft Ariane $5$ due to an unexpected conversion from a $64$-bit floating point number to a $16$-bit integer \cite{lions1996ariane} and an error with computing time with both integer and floating point numbers leading to a failure in the Patriot missile system leading to the death of $28$ people during the Gulf war \cite{skeel}. By now it is clear that there is an imperative need to devise techniques to address such issues. Methods for dealing with either correcting such errors, detecting them or re-writing code to mitigate such issues, broadly speaking, falls under static, dynamic and error bound determining approaches. Static analysis generally uses abstract interpretation to provide sound approximations of the semantics of programs such as in \cite{titolo2018abstract} and \cite{martel2017floating}. On the other hand, dynamic approaches involve either running a program several times to estimate error or inserting extra code to analyze certain computations as can be seen in \cite{lam2016fine}. Work related to finding relative and absolute error bounds can be found in \cite{darulova2014sound} and \cite{rubio2016floating}. Correction Linéaire Automatique des Erreurs d'Arrondi (CENA) is a unique correcting method introduced by Langlois in \cite{langlois2001automatic} that, instead of approximating or analyzing a program, computes the forward error accumulated over the program's execution. This paper analyzes the effectiveness of the CENA compensations on Newton's method for one-dimensional zero finding. In Section~\ref{sec:fpa} the standard model of floating point numbers and arithmetic is introduced. Section~\ref{sec:eft} presents error free transformations and how they are computed. Then, Section~\ref{sec:cena} presents the CENA method and ties everything from the previous section together. The framework for CENA Newton's method is given in Section~\ref{sec:newton} along with experimental results on a specific type of polynomial. Finally, in Section~\ref{sec:eval} the flaws in the analysis are outlined along with an hollistic overview of CENA and Section~\ref{sec:conc} provides a conclusion. 
\end{document}