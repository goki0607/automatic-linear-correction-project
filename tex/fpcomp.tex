\documentclass{standalone}

\begin{document}
	\subsection{The Standard Model of Representation}
	The IEEE Standard for Floating-Point Arithmetic \cite{zuras2008ieee}, or by its more common name IEEE-$754$, is the current standard for working in finite precision arithmetic and ``mandates that floating point operations be peformed as if the computation was done with infinite precision and then rounded.'' \cite{boldo2015verified} Let $\mathbb{F}\subset\mathbb{R}$ be the set of floating point numbers with precision $\epsilon_{M}$ where $\epsilon_{M}$ is commonly referred to as ``machine precision.'' A normalized number $\hat{x}\in\mathbb{F}$ is represented as$$\hat{x}=\pm(1+m)\times b^{e}$$where $b=2$ or $b=10$ is the base, $e$ (the exponent) is a signed integer and $m$ is a fixed point value between $0$ and $1$. In single precision floating point numbers $m$ is a $23$-bit value and $e$ is a $8$-bit value while in double precision floating point numbers $m$ is a $52$-bit value and $e$ is an $11$-bit value. Taking into the account the signed bit, there are $32$ bits for single and $64$ bits for double precision numbers. Special values occur when $e=0,m=0$ which gives $+0$ and $-0$ depending on the sign bit, $e=1\dots1,m=1$ which gives $+\infty$ and $-\infty$, and $e=1\dots1,m=0$ which gives the value ``not a number', also known as \texttt{NaN}. In general, operations that analytically lead to an indeterminate form, such as $\infty-\infty$, will produce \texttt{NaN}. Denormalzied (also known as subnormal) numbers $\tilde{x}\in\mathbb{F}$ in the form$$\tilde{x}=\pm(0+m)\times 2^{e}$$allow the system to represent small numbers with zero exponent and non-zero mantissa by allowing for gradual underflow. Without loss of generality, this project will assume the most commonly used floating point numbers which are the normalized values with base $b=2$. IEEE-$754$ also specifies other precisions and a nice summary table can be found in \cite{martel2017floating}.
	\subsection{Rounding}
	Referring back to \cite{boldo2015verified}, rounding modes that are given in \cite{zuras2008ieee} are:
	\begin{itemize}
		\item round towards nearest,
		\item round towards $+\infty$,
		\item round towards $-\infty$, and
		\item round towards $0$.
	\end{itemize}
	In the case of round towards nearest, the ``tablemaker's dilemma'' situation arises where a choice must be made on how to round a value that lies exactly in between two representable numbers. The two options for resolving the dilemma are:
	\begin{itemize}
		\item round to the nearest even, and
		\item round to the largest magnitude.
	\end{itemize}
	Combining everything together, let $x\in\mathbb{R}$ be some number and $\hat{x}\in\mathbb{F}$ be its rounded floating point representation. A standard way of expressing the transformation from $x$ to $\hat{x}$ is$$\hat{x}=fl(x)=x(1+\delta),\quad\text{with}\:\:\abs{\delta}<\epsilon_{M}$$where $\delta$ is the error associated with rounding and is zero if and only if $x$ is exactly represntable. Again, without loss of generality, this project will assume the most commonly used rounding of round towards nearest with rounding direction towards the nearest even where $\epsilon_{M}=2^{-24}$ for single-precision and $\epsilon_{M}=2^{-53}$ for double-precision floating point representations.
	\subsection{The Standard Model of Arithmetic}
	Elementary floating point operations are defined by the set $\{+,-,\times,/\}$. For $\hat{x},\hat{y}\in\mathbb{F}$ and $\circ\in\{+,-,\times,/\}$ let $z=\hat{x}\circ\hat{y}$ and $\hat{z}=fl(z)=fl(\hat{x}\circ\hat{y})$ (with $\hat{y}\ne0$ if $\circ=/$). Then, along with the standard model of representation and rounding defined above, the standard model of arithmetic is given by$$\hat{z}=fl(z)=fl(\hat{x}\circ\hat{y})=(\hat{x}\circ\hat{y})(1+\delta_{1}),\quad\text{with}\:\:\abs{\delta}\le\epsilon_{M},\quad\text{where}\:\:\circ\in\{+,-,\times,/\}$$where $\delta$ is the the error associated with the rounding of elementary operations, or simply elementary rounding errors, assuming no overflow or underflow has occurred. \cite{higham2002accuracy} ``The model says that the computed value of $\hat{x}\circ\hat{y}$ is ``as good as'' the rounded exact answer, in the sense that the relative error bound is the same in both cases.'' \cite{higham2002accuracy} Furthermore, the consideration of the errors associated with $\hat{x}$ and $\hat{y}$ due to different possible ``lengths'' of the values is not necessary due to the use of guard digits in the IEEE-$754$ standard. \cite{higham2002accuracy}
\end{document}